\documentclass[draft]{fhnwreport}       %[mode] = draft or final
                                        %{class} = fhnwreport, article, 
                                        %          report, book, beamer, standalone
\input{header}			                %loads all packages, definitions and settings												
\title{Kickoff Meeting Bachelor Thesis}          %Project Title
\author{Tobias Koller}          %Document Type => Technical Report, ...
\date{Online, 06.02.2020}             %Place and Date

\begin{document}
%%---TITLEPAGE---------------------------------------------------------------------------
\selectlanguage{english}                %ngerman or english
\maketitle

\vspace*{-1cm}						    %compensates the space after the date line.
\vfill

{
\renewcommand\arraystretch{2}
\begin{center}
\begin{tabular}{>{\bf}p{4cm} l}
Organization                  &    FHNW, School of Business\\
Study program                 &    Business Information Technology\\
Author   	                  &    Tobias Koller\\
Supervisor                    &    Prof. Dr. Kaspar Riesen\\
Customer                      &    Prof. Dr. Thomas Hanne\\
Expert                        &    René Pickhardt
\end{tabular}
\end{center}
}
\clearpage

\pagenumbering{arabic}
\section{Basics}
\subsection{Kommentare}\label{sec:todos}
Möchte man Kommentare setzen kann dies einfach mit dem Package \texttt{todonotes} realisiert werden. Die Kommentare können mit dem Befehl \verb|\todo{}| am Seitenrand oder für längere Kommentare mit \verb|\todo[inline]{}| direkt im Text angeordnet werden. \todo{Todo am Seitenrand}
\todo[inline]{Todo im Text}

%%---NOTES for DEBUG---------------------------------------------------------------------
\ifdraft{%Do this only if mode=draft
%%requires \usepackage{todonotes})
\newpage
\listoftodos[\section{Todo-Notes}]
\clearpage
}
{%Do this only if mode=final
}
\end{document}
