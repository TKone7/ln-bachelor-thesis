\documentclass[final]{fhnwreport}       %[mode] = draft or final
                                        %{class} = fhnwreport, article, 
                                        %          report, book, beamer, standalone
\usepackage{hyperref}
%%---Main Packages-----------------------------------------------------------------------
\usepackage[english, ngerman]{babel}	%Mul­tilin­gual sup­port for LaTeX
\usepackage[T1]{fontenc}				%Stan­dard pack­age for se­lect­ing font en­cod­ings
\usepackage[utf8]{inputenc}				%Ac­cept dif­fer­ent in­put en­cod­ings
\usepackage{lmodern}                    %The newer Font-Set
\usepackage{textcomp}					%LaTeX sup­port for the Text Com­pan­ion fonts
\usepackage{graphicx} 					%En­hanced sup­port for graph­ics
\usepackage{float}						%Im­proved in­ter­face for float­ing ob­jects
\usepackage{ifdraft}                    %Let you check if the doc is in draft mode

%%---Useful Packages---------------------------------------------------------------------
\usepackage[pdftex,dvipsnames]{xcolor}  %Driver-in­de­pen­dent color ex­ten­sions for LaTeX
\usepackage{csquotes}                   %Simpler quoting with \enquote{}
\usepackage{siunitx} 					%A com­pre­hen­sive (SI) units pack­age
\usepackage{listings}					%Type­set source code list­ings us­ing LaTeX
\usepackage[bottom]{footmisc}			%A range of foot­note op­tions
\usepackage{footnote}					%Im­prove on LaTeX's foot­note han­dling
\usepackage{verbatim}					%Reim­ple­men­ta­tion of and ex­ten­sions to LaTeX ver­ba­tim
\usepackage[textsize=footnotesize]{todonotes} %Mark­ing things to do in a LaTeX doc­u­ment

%%---Tikz Packages-----------------------------------------------------------------------
\usepackage{standalone}
\usepackage{tikz}
\usepackage{circuitikz}
\usetikzlibrary{arrows}
\usetikzlibrary{calc}
\usetikzlibrary{intersections}

%%---Math Packages-----------------------------------------------------------------------
\usepackage{amsmath}					%AMS math­e­mat­i­cal fa­cil­i­ties for LaTeX
%\usepackage{amssymb}					%Type­set­ting symbols (AMS style)
%\usepackage{array}						%Ex­tend­ing the ar­ray and tab­u­lar en­vi­ron­ments
%\usepackage{amsthm}					%Type­set­ting the­o­rems (AMS style)

%%---Table Packages----------------------------------------------------------------------
\usepackage{tabularx}					%Tab­u­lars with ad­justable-width columns
%\usepackage{longtable}
\usepackage{multirow}					%Create tab­u­lar cells span­ning mul­ti­ple rows
\usepackage{multicol}					%In­ter­mix sin­gle and mul­ti­ple columns

%%---PDF / Figure Packages---------------------------------------------------------------
\usepackage{pdfpages}					%In­clude PDF doc­u­ments in LaTeX
\usepackage{pdflscape}					%Make land­scape pages dis­play as land­scape
\usepackage{subfig}					    %Fig­ures di­vided into sub­fig­ures

%%---Other Packages----------------------------------------------------------------------
%\usepackage{xargs}                     %De­fine com­mands with many op­tional ar­gu­ments

%%---Bibliography------------------------------------------------------------------------
\usepackage[style=ieee,urldate=comp,backend=biber]{biblatex}
\addbibresource{literature/bibliography.bib}

%%---Main Settings-----------------------------------------------------------------------
\graphicspath{{./graphics/}}			%Defines the graphicspath
\geometry{twoside=false}				    %twoside=false disables the "bookstyle"
\setlength{\marginparwidth}{2cm}
\overfullrule=5em						%Creates a black rule if text goes over the margins => debugging




%%---User Definitions--------------------------------------------------------------------
%%Tabel-Definitions: (requires \usepackage{tabularx})
\newcolumntype{L}[1]{>{\raggedright\arraybackslash}p{#1}}    %column-width and alignment
\newcolumntype{C}[1]{>{\centering\arraybackslash}p{#1}}
\newcolumntype{R}[1]{>{\raggedleft\arraybackslash}p{#1}}

%%---Optional Package Settings-----------------------------------------------------------
%Listings-Settings: (requires \usepackage{listings}) => Example with Matlab Code
\lstset{language=Matlab,%
    basicstyle=\footnotesize\ttfamily,
    breaklines=false,%
    morekeywords={switch, case, otherwise},
    keywordstyle=\color{Blue},%
    tabsize=2,
    %morekeywords=[2]{1}, keywordstyle=[2]{\color{black}},
    identifierstyle=\color{Black},%
    stringstyle=\color{Purple},
    commentstyle=\color{Green},%
    showstringspaces=false,%without this there will be a symbol in the places where there is a space
    numbers=left,%
    numberstyle={\tiny \color{black}},% size of the numbers
    numbersep=9pt, % this defines how far the numbers are from the text
    %emph=[1]{word1, word2,...},emphstyle=[1]\color{red}
}										                %loads all packages, definitions and settings												
\title{Kickoff Meeting Bachelor Thesis}          %Project Title
\author{René Pickhardt, Tobias Koller}          %Document Type => Technical Report, ...
\date{Online, 06.02.2020}             %Place and Date

\begin{document}
%%---TITLEPAGE---------------------------------------------------------------------------
\selectlanguage{english}                %ngerman or english
\maketitle

\vspace*{-1cm}						    %compensates the space after the date line.
\vfill

{
\renewcommand\arraystretch{2}
\begin{center}
\begin{tabular}{>{\bf}p{4cm} l}
Organization                  &    FHNW, School of Business\\
Study program                 &    Business Information Technology\\
Author   	                  &    Tobias Koller\\
Supervisor                    &    Prof. Dr. Kaspar Riesen\\
Project sponsor               &    Prof. Dr. Thomas Hanne\\
Expert                        &    René Pickhardt
\end{tabular}
\end{center}
}
\clearpage

\pagenumbering{arabic}
\section{Introduction}
René Pickhardt and Tobias Koller met on February 6, online to hold the kickoff meeting in a Skype session. The main goals were to get acquainted with each other, clearify any uncertainties from the previous paper upon which the thesis will build and to define the topic.

René Pickhardt is a data scientist and lightning network protocol developer. Currently he is a PhD candidate at the Norwegian University of Science and Technology, NTNU. He is the co-author of the publication ``Imbalance measure and proactive channel rebalancing algorithm for the Lightning Network'' which serves as a starting point for the problem description.
\todo[inline]{Bitte mir melden, falls du zu deiner Vorstellung noch etwas anfügen / ändern möchtest.}



\section{Form of Collaboration}
While the project sponsor is represented by Prof. Dr. Thomas Hanne at FHNW, René Pickhardt has profound knowledge of the Lightning protocol and, therefore,  will consult Tobias Koller on all questions regarding the Bitcoin and Lightning Network. Because of different locations, communication with René Pickhardt will be mainly via e-mail and, when required, video calls. Physical meetings are not required. René Pickhardt will be invited to all intermediate or final presentations whereas participation is voluntarily.

Tobias Koller agrees to refine the findings from his thesis and, together with René Pickhardt, compile a publication for the community once the work is completed.  

\section{Core Objectives}
\subsection{Starting point}
René Pickhardt's and Mariusz Nowostawski's publication ``Imbalance measure and proactive channel rebalancing algorithm for the Lightning Network'' (\url{https://arxiv.org/abs/1912.09555}) serves as a base to formulate the question for this thesis. In their work they present a solution for the path finding problem in a privacy-aware payment channel network. In such a network the sender of a payment needs to calculate a path of payment channels to the recipient. While the capacities of channels are public information the local distribution between the two channel parties are private which makes finding a path a difficult task. The proposed solution includes a rebalancing protocol which the nodes of the network should follow to achieve a higher balancedness (for itself but also for the entire network). It consists of instructions to proactively rebalance their channels within their friend of a friend's network, redistributing the relative funds owned in a channel but leaving total funds owned unchanged.

However, these payment channel networks are decentralized by nature and no protocol changes can be forced upon the node operators. Therefore, the question arises how effective this protocol change will be assuming only partial participation of nodes. What are the effects of different levels of participation on the imbalance measure \footnote{Defined as the inequality of the distribution of a nodes channel balance coefficients} of the network during repeated rebalancing cycles? What is the effect of different levels of participation on the networks ability to route payments between random nodes?

\subsection{Experiment}
In order to be able to compare the results of partial participation with the previous findings the experiment should be repeated in a similar way. The main independent variable should be the relative participation of nodes (in \%) in the new protocol. The dependent variables to test should then be the network's imbalance measure represented by the Ginicoefficients as well as different measures to determine the networks ability to route payments. 

\subsection{Findings}

\todo[inline]{Ich könnte hier schon konkrete Diagramme sketchen wie du vorgeschlagen hast. Allerdings möchte ich dies gerne fürs Proposal aufsparen (nächster step)}
The results from the experiment could show that the networks ability to route payments increase linearly with the degree of node participation. Alternatively, a small participation might lead to a unproportional large improvement in routing capabilities or the other way around. These answers will help the Lightning protocol developers to make decisions about how to implement such changes.

%%---NOTES for DEBUG---------------------------------------------------------------------
\ifdraft{%Do this only if mode=draft
%%requires \usepackage{todonotes})
\newpage
\listoftodos[\section{Todo-Notes}]

\clearpage
}
{%Do this only if mode=final
}
\end{document}
